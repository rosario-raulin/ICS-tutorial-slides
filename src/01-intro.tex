\documentclass[12pt]{beamer}

\usepackage[utf8]{inputenc}
\usepackage{graphicx}
\usepackage{wrapfig}
\usepackage{url}

\usetheme{Rochester}

\title{Tutorium - Einführung in die Informatik}
\author{Kilian Gärtner, Rosario Raulin}
\date{Wintersemester 2012/13}

\begin{document}

\frame{\titlepage}

\section{Wer sind wir?}

\begin{frame}

	\frametitle{Wer wir sind (1)}

	\begin{itemize}
		\item Rosario Raulin
		\item Informatik-Student im 3. Semester
		\item Studienprofil: Webgründer
		\item E-Mail: raulin@st.ovgu.de
		\item Facebook, Twitter, ...
	\end{itemize}

	\begin{flushright}
		\includegraphics[scale=0.1]{src/img/babygnu}
	\end{flushright}

\end{frame}

\begin{frame}

	\frametitle{Wer wir sind (2)}

	\begin{itemize}
		\item Kilian Gärtner
		\item Informatik-Student im 3. Semester
		\item E-Mail: kilian.gaertner@st.ovgu.de
		\item Facebook, Minestar.de, ...
	\end{itemize}

	\begin{flushright}
		\includegraphics[scale=0.55555]{src/img/wurm}
	\end{flushright}

\end{frame}

\begin{frame}
	\frametitle{Agenda}
	\tableofcontents
\end{frame}


\section{Ziel und Motivation}

\begin{frame}

\frametitle{Worum geht's eigentlich?}

	\begin{quote}
		Ein Tutorium oder Tutorat ist an einer Hochschule eine Lehrveranstaltung
		im Grundstudium, in der ein fortgeschrittener Student eine
		Lehrveranstaltung unterstützt, indem er mit den Teilnehmern
		Grundkenntnisse vertieft und -fertigkeiten einübt.

		\begin{flushright}
			\scriptsize Quelle: Wikipedia
		\end{flushright}

	\end{quote}
	
\end{frame}

\begin{frame}

	\frametitle{Oder anders ausgedrückt...}

	\pause
	\begin{center}
		\includegraphics[scale=0.42]{src/img/help}
	\end{center}
	\pause

	\begin{itemize}
		\item wöchentlich, mittwochs, 11.15 - 12.45 Uhr, G29-336
		\item alle 2 Wochen "Pflicht" (wir zwingen Euch aber nicht)
		\item \url{http://tutorium.rosario-raulin.de}
	\end{itemize}

\end{frame}

\begin{frame}

	\frametitle{Unsere Ziele}

	\begin{wrapfigure}{l}{60px}
		\includegraphics[scale=0.15]{src/img/doc}
	\end{wrapfigure}

	Unsere Ziele:
	\begin{enumerate}
		\item Abbrecherquote senken
		\item sehr gute Prüfungsergebnisse
		\item Programmierspaß wecken
	\end{enumerate}

\end{frame}

\begin{frame}

	\frametitle{Warum sollte ich mitmachen?}

	\begin{wrapfigure}{r}{0pt}
		\includegraphics[scale=0.55555]{src/img/tools}
	\end{wrapfigure}


	Weil...
	\pause
	\begin{itemize}
		\item du eine Frage hast
		\pause
		\item du etwas nicht verstehst
		\pause
		\item du nicht weiß, warum X wichtig ist
		\pause
		\item du programmieren möchtest
		\pause
		\item der Wettbewerb im Sommersemester kommt
	\end{itemize}

\end{frame}

\begin{frame}
	\frametitle{Take-Home-Message}
	\begin{quote}
		Wenn man etwas nicht weiß, so kann man fragen; wenn man etwas nicht
		kann, so kann man es lernen.
		\newline
		\begin{flushright}
		\scriptsize Lü Buwei, chinesischer Kaufmann
		\end{flushright}
	\end{quote}

\end{frame}

\section{Vorstellungsrunde}
\section{Ausgabe der FIN-Accounts}
\section{Einführung in Eclipse}
\section{Fragen und Antworten}

\begin{frame}
	\frametitle{Vorstellungsrunde}
	\begin{enumerate}
		\item Wie heißt du?
		\item Wie alt bist du?
		\item Was studierst du?
		\item Warum studierst du das?
		\item Was sind dein Hobbys?
		\item Hast du schon Programmiererfahrung?
		\item Welche Wünsche für das Tutorium hast du?
	\end{enumerate}
\end{frame}

\end{document}

